%% Generated by Sphinx.
\def\sphinxdocclass{report}
\documentclass[letterpaper,10pt,english]{sphinxmanual}
\ifdefined\pdfpxdimen
   \let\sphinxpxdimen\pdfpxdimen\else\newdimen\sphinxpxdimen
\fi \sphinxpxdimen=.75bp\relax

\PassOptionsToPackage{warn}{textcomp}
\usepackage[utf8]{inputenc}
\ifdefined\DeclareUnicodeCharacter
% support both utf8 and utf8x syntaxes
\edef\sphinxdqmaybe{\ifdefined\DeclareUnicodeCharacterAsOptional\string"\fi}
  \DeclareUnicodeCharacter{\sphinxdqmaybe00A0}{\nobreakspace}
  \DeclareUnicodeCharacter{\sphinxdqmaybe2500}{\sphinxunichar{2500}}
  \DeclareUnicodeCharacter{\sphinxdqmaybe2502}{\sphinxunichar{2502}}
  \DeclareUnicodeCharacter{\sphinxdqmaybe2514}{\sphinxunichar{2514}}
  \DeclareUnicodeCharacter{\sphinxdqmaybe251C}{\sphinxunichar{251C}}
  \DeclareUnicodeCharacter{\sphinxdqmaybe2572}{\textbackslash}
\fi
\usepackage{cmap}
\usepackage[T1]{fontenc}
\usepackage{amsmath,amssymb,amstext}
\usepackage{babel}
\usepackage{times}
\usepackage[Bjarne]{fncychap}
\usepackage{sphinx}

\fvset{fontsize=\small}
\usepackage{geometry}

% Include hyperref last.
\usepackage{hyperref}
% Fix anchor placement for figures with captions.
\usepackage{hypcap}% it must be loaded after hyperref.
% Set up styles of URL: it should be placed after hyperref.
\urlstyle{same}
\addto\captionsenglish{\renewcommand{\contentsname}{Contents:}}

\addto\captionsenglish{\renewcommand{\figurename}{Fig.}}
\addto\captionsenglish{\renewcommand{\tablename}{Table}}
\addto\captionsenglish{\renewcommand{\literalblockname}{Listing}}

\addto\captionsenglish{\renewcommand{\literalblockcontinuedname}{continued from previous page}}
\addto\captionsenglish{\renewcommand{\literalblockcontinuesname}{continues on next page}}
\addto\captionsenglish{\renewcommand{\sphinxnonalphabeticalgroupname}{Non-alphabetical}}
\addto\captionsenglish{\renewcommand{\sphinxsymbolsname}{Symbols}}
\addto\captionsenglish{\renewcommand{\sphinxnumbersname}{Numbers}}

\addto\extrasenglish{\def\pageautorefname{page}}

\setcounter{tocdepth}{1}



\title{simulation Documentation}
\date{Oct 17, 2018}
\release{}
\author{f}
\newcommand{\sphinxlogo}{\vbox{}}
\renewcommand{\releasename}{}
\makeindex
\begin{document}

\pagestyle{empty}
\maketitle
\pagestyle{plain}
\sphinxtableofcontents
\pagestyle{normal}
\phantomsection\label{\detokenize{index::doc}}

\phantomsection\label{\detokenize{index:module-propagation_time}}\index{propagation\_time (module)}\index{generate\_transaction\_time\_distribution() (in module propagation\_time)}

\begin{fulllineitems}
\phantomsection\label{\detokenize{index:propagation_time.generate_transaction_time_distribution}}\pysiglinewithargsret{\sphinxcode{\sphinxupquote{propagation\_time.}}\sphinxbfcode{\sphinxupquote{generate\_transaction\_time\_distribution}}}{\emph{N: int}, \emph{p: int}, \emph{peer\_iterations: int}, \emph{start\_node\_iterations: int}}{}
Generates and saves transaction relay times for dispering a transaction to the network. Saved to .mat file.
\begin{description}
\item[{N}] \leavevmode{[}int{]}
Number of nodes in network

\item[{p}] \leavevmode{[}str{]}
Number of peers for each node

\item[{peer\_iterations}] \leavevmode{[}int{]}
number of different peer distributions to iterate simulation over. !!!Should be moved outside of function and each run saved seperately!!

\item[{start\_node\_iterations}] \leavevmode{[}int{]}
number of different starting nodes to iterate simulation over. !!!Should be moved outside of function and each run saved seperately!!

\end{description}

\end{fulllineitems}

\index{relay\_single\_transaction() (in module propagation\_time)}

\begin{fulllineitems}
\phantomsection\label{\detokenize{index:propagation_time.relay_single_transaction}}\pysiglinewithargsret{\sphinxcode{\sphinxupquote{propagation\_time.}}\sphinxbfcode{\sphinxupquote{relay\_single\_transaction}}}{\emph{peers}, \emph{start\_node}}{}
Relays a transaction through the peer network.

as nodes recieve the transaction, their peers are added to the upcoming\_events collection
to allow the transaction to be relayed in the correct order. The transaction is relayed until all nodes 
have recieved it.
\begin{description}
\item[{peers}] \leavevmode{[}array{]}
Array containing the peers of each node

\item[{start\_node}] \leavevmode{[}int{]}
start\_node is the node where the transaction originates

\end{description}
\begin{description}
\item[{float array}] \leavevmode
Array of times taken for nodes to recieve transaction. Node number is index, value is time taken.

\end{description}

\end{fulllineitems}



\chapter{Indices and tables}
\label{\detokenize{index:indices-and-tables}}\begin{itemize}
\item {} 
\DUrole{xref,std,std-ref}{genindex}

\item {} 
\DUrole{xref,std,std-ref}{modindex}

\item {} 
\DUrole{xref,std,std-ref}{search}

\end{itemize}


\renewcommand{\indexname}{Python Module Index}
\begin{sphinxtheindex}
\let\bigletter\sphinxstyleindexlettergroup
\bigletter{p}
\item\relax\sphinxstyleindexentry{propagation\_time}\sphinxstyleindexpageref{index:\detokenize{module-propagation_time}}
\end{sphinxtheindex}

\renewcommand{\indexname}{Index}
\printindex
\end{document}